\chapter{Grundlagen}
Das sind die Grundlagen

\chapter{Definitionen(?)}

\section{GeoMultiSens}
Beschreibung von GeoMultiSens, GfZ, 

\section{Python}
Python ist eine quelloffene und universell einsetzbare Programmiersprache, die seit 1989 existiert und fortwährend weiter entwickelt wird. Prägende Eigenschaften der Sprache sind unter anderem eine dynamische Typisierung von Variabeln, eine simpel gehaltene Syntax und die Erweiterbarkeit durch Module und Bibliotheken. Es ist auch möglich Python-Code durch C- beziehungsweise C++-Bibliotheken zu erweitern \cite{Martelli2006}. Dies ermöglicht eine verkürzte Ausführungszeit eines Programms, insbesondere bei rechenintensiven Programmabschnitten. Ein Schwachpunkt von Python im Bezug auf die schnelle Verarbeitung großer Datenmengen ist die nicht auf automatisierte Parallelisierung ausgelegte \textcolor{blue}{Struktur}.

\section{Java}

\chapter{Evaluation}
Beschreibung und Bewertung der Ergebnisse meiner Untersuchungen

\chapter{Fazit}
Fazit und Ausblick

