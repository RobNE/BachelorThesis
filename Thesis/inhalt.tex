\pagenumbering{arabic}
\chapter{Einleitung}
In den vergangenen Jahren war ein massiver Zunahme des generierten Datenaufkommens zu beobachten \cite{EMC2014}. Viele Projekte, Unternehmen und Institutionen haben Zugriff auf eine gewaltige Menge an Daten. Diese wächst immer schneller an. 2004 analysierte Google circa 100 Terabyte pro Tag \cite{Dean2004}. Bis zum Jahr 2008 war die täglich zu analysierende Datenmenge bereits auf 20 Petabyte angewachsen \cite{Dean2008}. Das Sloan Digital Sky Survey, das ein Viertel des Himmels astronomisch erkundet, hat seit 1998 insgesamt 116 Terabyte an astronomischen Daten gesammelt \cite{York2000, Alam2015}. Jede Nacht kommen circa 250 Gigabyte neu hinzu \textcolor{blue}{Quelle: Herr Prof. Freytags VL, dort angegebene Quelle ist offline. Wie angeben?}. Ein weiteres Beispiel ist das 1000 Genomes Project \cite{Baker2010}, das zwischen 2008 und 2013 insgesamt 464 Terabyte Daten zum menschlichen Genom sammelte. Insgesamt werden die Datenmengen weiter stark zunehmen, für das Jahr 2020 wird eine weltweites Datenaufkommen von 44 Zettabyte prognostiziert \cite{EMC2014}. 
Diese Entwicklung offenbart diverse neue Herausforderungen bei der Speicherung, Verarbeitung und Analyse von Daten. Dabei spielt die möglichst schnelle Verarbeitung von stetig generierten Daten eine große Rolle. Diese muss im Gegensatz zur Verarbeitung bereits gespeicherter Daten abhängig vom aktuellen Datenaufkommen skalieren. Aktuelle Datenverarbeitungssysteme wie Apache Hadoop \cite{HadoopWebsite} und Apache Flink \cite{FlinkWebsite} bieten diese Möglichkeit der Datenflussanalyse und ermöglichen eine flexible Analyse der Daten. Kern dieser Systeme ist eine Implementierung des Map-Reduce Paradigmas \cite{Dean2008} sowie die Nutzung von User Defined Functions. Diese ermöglichen eine parallele Abarbeitung von Arbeitsschritten in einem direkten, azyklischen Graphen. Der DAG wird zuvor aus dem vom User bereitgestellten Quellcode erzeugt. Die durch diese Architektur erreichbare, massiv parallelisierbare Ausführung der Datenanalyse ermöglicht die Nutzung von Clustern. Somit wird eine skalierbare Infrastruktur genutzt, die wiederum eine skalierte Nutzung der Datenverarbeitungssysteme ermöglicht. Diese Systeme können auch die weiterhin wichtige Nutzung und Analyse von bereits konsistent gespeicherten Daten unter Nutzung des parallelen Verarbeitungsansatzes durchführen. Dies ist insbesondere deshalb notwendig, da traditionelle Datenbanksysteme große Datenmengen nicht immer in akzeptabler Form und Verarbeitungszeit verarbeiten können \cite{Jacobs2009}.
\newline
Das Hauptproblem bei der Verarbeitung von großen Datenmengen auf einzelnen Maschinen entsteht, wenn die zu verarbeitende Datenmenge die Hauptspeichergröße übersteigt. In diesem Fall müssen die nicht in den Hauptspeicher speicherbaren Daten zur Verarbeitungszeit nachgeladen werden, was die Verarbeitungszeit aufgrund der unterschiedlichen Beschaffenheit der verschiedenen Speicherebenen extrem verlängert. Um diese "Speicherklippe" zu umgehen, werden zunehmend parallelisierbare Ansätze der Datenverarbeitung verfolgt.
Im Rahmen dieser Bachelorarbeit sollen demzufolge ein traditioneller und ein massiv parallelisierbarer Ansatz bei der Verarbeitung von großen Datenmengen untersucht werden. So soll eine Abschätzung der Leistungsfähigkeit, Vorteile und Nachteile beider Ansätze ermittelt werden. Der Vergleich beider Ansätze wird am Beispiel eines Algorithmus zur Approximierung von Pixelzeitreihen durchgeführt. Dieser wird im Rahmen des Projekts GeoMultiSens\cite{GeoMultiSensWebsite} zur Analyse der Veränderung der Flora in einer geographischen Region genutzt. Dabei werden durch Landsat-Satelliten Satellitenaufnahmen bereitgestellt, die nach der Aufbereitung durch vorgestellte Algorithmen ausschnittweise untersucht werden. Nach der Analyse werden anschließend mithilfe des Algorithmus auf Basis der approximierten Werte Prognosen zur weiteren Entwicklung der Flora der untersuchten Region gestellt. Dabei werden bei einer Analyse mehrere Szenenausschnitte derselben geographischen Region analysiert. Dabei müssen große Datenmengen verarbeitet werden, so dass sich die Nutzung eines aktuellen Datenverarbeitungssystems anbietet. Bei dieser Bachelorarbeit wird als Vertreter der massiv parallelisierbaren Datenverarbeitungssysteme Apache Flink genutzt.
\newline
Es werden drei unterschiedliche Implementierungen des Algorithmus untersucht, die sich hinsichtlich der eingesetzten Technologien und Programmiersprachen unterscheiden. Die zugrunde liegende Methodik, die der Algorithmus implementiert, ist bei allen untersuchten Varianten identisch. Als Basis wird die bereits implementierte und in der Praxis genutzte Python-Implementation genutzt. Sie sollte somit die untere Schranke der Leistungsmessungen darstellen. Die zweite und dritte Variante werden in Flink implementiert. Diese beiden Varianten unterscheiden sich bezüglich der genutzten Programmiersprache. \textcolor{blue}{Zur Implementierung der zweiten Variante wird Java-Schnittstelle von Flink genutzt, zur Umsetzung von Variante drei die Python-Schnittstelle.} Schließlich werden alle drei Varianten unter identischen Bedingungen getestet. Dies bedeutet, dass sowohl die Testumgebung als auch die Testdaten identisch sein sollen. Dabei sollen alle Varianten sowohl auf einer leistungsfähigen Einzelmaschine als auch mit einem Cluster von Maschinen getestet werden. Ausgehend von den Untersuchungen und den ermittelten Ergebnissen soll nachfolgend eine Bewertung der drei Implementierungsvarianten des Algorithmus vorgenommen werden. Dabei sollen insbesondere die Größe der Ausgangsdatenmenge, die genutzte Hardware sowie die Größe der untersuchten Bildausschnitte in Bezug zu den Ergebnissen gesetzt werden.

%----------------------------------------------------------------------------------

\chapter{Grundlagen}
\section{Grundlagen der Satellitenbildanalyse}
\subsection{Fernerkundung mithilfe des Landsat-Satellitensystems}

Als Fernerkundung wird \textcquote{DIN18716}{die Gesamtheit der Verfahren zur Gewinnung von Informationen über die Erdoberfläche oder anderer nicht direkt zugänglicher Objekte durch Messung und Interpretation der von ihr ausgehenden (Energie-) Felder} verstanden. Fernerkundungssatelliten verfügen über verschiedene Aufnahmesysteme, die durch multispektrale Messungen von emittierter elektromagnetischer Strahlung eine berührungsfreie Beobachtung der Erdoberfläche ermöglichen. Bei der multispektralen Messung werden von Sensoren registrierte spektrale Signaturen einzelnen Bereichen des elektromagnetischen Spektrums zugeordnet. Das Resultat sind mehrere spektrumsspezifische, simultan aufgenommene Satellitenbilder, die nur das aufgefangene Licht eines spezifischen Spektralbereichs, auch Spektralband genannt, zeigen. Die Art und Qualität der Aufnahmesensoren ist dabei abhängig vom Typ des Satelliten. 

Die Ausgangsdaten für die Untersuchungen in dieser Bachelorarbeit wurden von Satelliten des Landsat-Satellitensystems aufgenommen. Der erste Landsat-Satellit Landsat 1 wurde 1972 gestartet. Seitdem wurden die Sensoren und die Satelliten kontinuierlich weiterentwickelt. Aktuell sind Landsat 7 und, im Rahmen der Landsat Data Continuity Mission, Landsat 8 im Einsatz. Landsat 8 nutzt in der aktuellen Generation zwei verschiedene Instrumente zur Fernerkundung. Den Operational Land Imager (OLI) und die Thermal Infrared Sensors (TIRS). 

Der OLI erfasst emittierte elektromagnetische Strahlung im Spektralbereich von 0,433 µm bis 1,390 µm unterteilt in acht Spektralkanäle sowie einen panchromatischen Kanal. Es werden mehr als 7000 Detektoren pro Spektralband genutzt, um eine bessere Bildqualität zu bieten als frühere Systeme \cite{Markham2004}. Neben den klassischen Farbspektren Blau, Grün und Rot nutzt Landsat-8 ein weiteres Band, das speziell für die Fernerkundung von Küsten genutzt wird. Außerdem verfügt Landsat-8 über drei Infrarotbänder, die nahes und mittleres Infrarotlicht registrieren, sowie ein weiteres Infrarotband, das auf die Beobachtung von Cirruswolken spezialisiert ist. Der panchromatische Kanal registriert elektromagnetische Strahlung mit Wellenlängen von 0,500 µm bis 0,680 µm. Dieser Spektralbereich entspricht etwa dem des menschlichen Auges. Aufgrund des, im Vergleich zu den einzelnen Farbfrequenzbändern, breiten abgedeckten Spektralbereichs ist eine höhere Auflösung der Bilder möglich.

Die Thermal Infrared Sensors (TIRS) \cite{Chaudhary2011} umfassen zwei Thermalkanäle. Diese erfassen im Gegensatz zu den Multispektralkanälen elektromagnetische Emissionen mit Wellenlängen zwischen 10,30 µm und 12,50 µm, also langwellige Infrarotstrahlung. Dies ist insbesondere für die Beobachtung von Wolken nützlich. Die Kantenlänge der einzelnen Pixel beträgt 100 Meter. Diese kann nachträglich auf 30 Meter angeglichen werden, um eine bessere Kompatibilität mit den Aufnahmen der Multispektralbänder zu gewährleisten.

Landsat 8 sendet pro Tag 400 Aufnahmen der Erdoberfläche, auch Szenen genannt, an die Bodenstation. Eine Aufnahme zeigt dabei eine geographische Region der Erde mit einer Ost-West-Ausdehnung von 185 Kilometer. Dies entspricht 100 nautischen Meilen. Die Nord-Süd-Ausdehnung einer Szene beträgt circa 174 Kilometer 

Durchschnittlich wird jede Region der Erde alle 16 (?) Tage überflogen \cite{Irons2012}.

Die von Landsat-Satelliten aufgezeichneten und übermittelten Bilder müssen jedoch vor der Durchführung von Analysen aufbereitet werden.

%----------------------------------------------------------------------------------

\subsection{Aufbereitung und Analyse von Satellitenbildern}
Die durch die Landsat-Satelliten aufgezeichneten und an die Bodenstationen übermittelten Szenen müssen vor ihrer Nutzung aufbereitet werden. Dadurch wird im Allgemeinen die Bildqualität verbessert, da externe Störfaktoren und eventuelle interne Fehlfunktionen ausgeglichen werden können. Es wird zwischen radiometrischen und die geometrischen Aufbereitungen unterschieden. Bei der radiometrischen Aufarbeitung werden digitale Werte wie zum Beispiel die Helligkeit der Szene angepasst. Eventuelle durch die Atmosphäre verursachte Verschlechterungen sollen verbessert werden, um ein genaueres Satellitenbild zu erhalten. Techniken um diese Verbesserung zu erreichen sind beispielsweise das Strahlungstransfermodell, die bildbasierte atmosphärische Korrektur und die Histogramm-Minimum-Methode. Es ist individuell von der Szene und den zur Verfügung stehenden Metadaten abhängig, mit welcher Methode die nützlichste Verbesserung erreicht werden kann.

Im Rahmen der geometrischen Aufbereitung sollen die Folgen einer eventuellen Fehlpositionierung des Satelliten korrigiert werden. Um die Szenen sinnvoll analysieren zu können, müssen sie korrekt und genau positioniert sein. Dies gilt insbesondere bei der Analyse mehrerer Szenen derselben geographischen Gegend. Um eine normierte Positionierung der Szenen zu schaffen, werden aus jeder Szene, die einen Teil der zu analysierenden geographischen Region beinhaltet, quadratische Teile der Originalszene ausgeschnitten. Dann wird jeder Pixel der Kachel auf die Zugehörigkeit zum Zielgebiet geprüft. Wenn ein Pixel relevant ist, wird er anhand seiner, aus der Position des Satelliten zum Aufnahmezeitpunkt ermittelten, Position in einem finalen Bild hinzugefügt. 


 
Durch die zunehmend bessere Qualität von Satellitenbildern, die durch Fernerkundungsatelliten aufgezeichnet werden \cite{Markham2004}, können detailliertere Analysen getätigt werden. Jedoch steigt mit zunehmender Größe der Bilddateien auch der Rechenaufwand, um die Szenen aufzubereiten und zu analysieren. Mit zunehmender Datenmenge wird eine massiv parallelisierbare Vorgehensweise bei der Aufbereitung und der Analyse von Satellitenbildern attraktiver, da sie besser skalierbar ist.

%----------------------------------------------------------------------------------

\section{Parallele Datenverarbeitungssysteme}
Um die seit mehreren Jahren massiv ansteigenden Datenmengen \cite{EMC2014} zu verarbeiten, wird eine verteilte Verarbeitung dieser Daten zunehmend populär. Dazu werden mehrere Maschinen zu einem Netzwerk, einem sogenannten Cluster, zusammengeschlossen. Diese Computer wären als einzelne Maschine nicht in der Lage ein großes beziehungsweise komplexes Problem in akzeptabler Zeit zu lösen. Die Leistungsfähigkeit des Netzwerks wird jedoch nicht über die Leistung einer einzelnen Maschine sondern primär über die Menge der zusammengeschlossenen Computer gesteuert. Dies hat mehrere Vorteile gegenüber der Verarbeitung mithilfe einzelner, besonders leistungsstarker Maschinen. Die wichtigsten Vorteile parallelisierter Systeme sind ihre Skalierbarkeit sowie die Fehlertoleranz. Falls mehr Rechenleistung benötigt wird oder wenn Teile des Netzwerks nicht funktionsfähig sind lassen sich neue Maschinen kurzfristig, meist auch im laufenden Betrieb, in das bestehende Netzwerk integrieren. Bei einzelnen, sehr leistungsstarken Computern gestaltet sich beides aufgrund der abgeschlossenen Beschaffenheit der Maschine schwierig. [Quelle]

Eine Big-Data Anwendung zeichnet sich durch drei Eigenschaften aus. Diese drei Charakteristika sind die Größe (engl. Volume), die Komplexität (eng. Variety) und die echtzeitnahe Verfügbarkeit sowie schnelle Verarbeitung (engl. Velocity) der Daten \cite{Laney2001}. Eine weiteres Merkmal ist die nicht garantierte Zuverlässigkeit und Einheitlichkeit der Daten (engl. veracity) \cite{Zikopoulos2012}. 
In den letzten Jahren hat sich das global produzierte Datenaufkommen massiv gesteigert. Insbesondere die zunehmende Zahl der Internetnutzer sowie die Verbreitung von Smartphones trägt zu dieser Entwicklung bei. Ebenso trägt die zunehmende Digitalisierung der Industrie sowie die zunehmende Verbreitung von Sensoren jeglicher Art zu diesem Anstieg bei. Aber auch in nicht kommerziellen Bereichen wächst die Datenmenge. Der Satellit Landsat 8 produziert Aufnahmen, die dreimal soviel Speicherplatz benötigen wie die der vorigen Generation. Wissenschaftliche Projekte produzieren fortwährend große Datenmengen. Beispielsweise das Sloan Digital Sky Survey, das seit 1998 insgesamt 116 Terabyte an astronomischen Daten gesammelt hat \cite{York2000, Alam2015}. Ein weiteres Beispiel ist das 1000 Genomes Project \cite{Baker2010}, das zwischen 2008 und 2013 insgesamt 464 Terabyte Daten zum menschlichen Genom sammelte. Insgesamt werden die Datenmengen weiter massiv zunehmen. Für das Jahr 2020 wird eine weltweites Datenaufkommen von 44 Zettabyte prognostiziert \cite{EMC2014}. Aus dieser steigenden Datenmenge ergeben sich auch Folgen für Daten verarbeitende Dienste. Es müssen sehr viel mehr Daten auf einmal verarbeitet werden. Darüber hinaus sind die zu verarbeitenden Daten zunehmend vielfältiger und unstrukturierter. Die verarbeitenden Algorithmen und die Speicherstrukturen müssen also hinreichend auf unvollständige beziehungsweise fehlerhafte Datensätze reagieren können und diese trotzdem bestmöglich verarbeiten. [Quellen] [Beschreibung für velocity einfügen].
Wenn eine Anwendung eine Datenmenge verarbeitet, die mindestens einige der vier Kriterien nach \cite{Laney2001} erfüllt, gilt diese Anwendung als Big-Data Anwendung. 


%Eigenschaften von: Big Data, DBMS, Grundlagen für Flink, Erwähnung MapReduce Prinzip

%----------------------------------------------------------------------------------

\section{Programmierabstraktionen}
\subsection{Apache Flink}
%Eigenschaften + Operatoren in Flink

Apache Flink ist eine Plattform, die auf eine massiv parallelisierte Verarbeitung von großen Datenmengen spezialisiert ist. Sie ging 2014 [Quelle] aus Stratosphere hervor, das seit 2010 kooperativ von Forschern verschiedener Universitäten entwickelt wurde \cite{Battre2010, Alexandrov2014}. Seit Januar 2015 ist Apache Flink ein Top-Level Projekt der Apache Software Foundation \cite{ApacheFlinkBlogEntry}. 

Die Hauptkomponenten der Plattform sind die Flink-Laufzeitumgebung und der Flink-Optimierer. Darüber hinaus gibt es Flink-Operatoren, die auf einen Datensatz angewendet werden können. Der Flink-Optimierer erhält einen azyklischen Graphen von Flink-Operatoren als Eingabe. Dieser wird mithilfe von Techniken der traditionellen Optimierung von relationalen Anfragen vereinfacht. Der optimierte Datenflussgraph, auch Jobgraph genannt, besteht aus mehreren, teilweise unabhängig voneinander zu verarbeitenden Arbeitsschritten. Diese können teilweise parallel bearbeitet werden. Die effiziente Organisation und Ausführung der einzelnen Arbeitsschritte ist die Aufgabe der Flink-Laufzeitumgebung. Sie verteilt die einzelnen Arbeitsschritte gemäß ihrer Abfolge im DAG an die einzelnen Knoten des Netzwerkes, auf dem die Flink-Anwendung ausgeführt wird.

%Aufbau: MasterNode, WorkerNode, MemoryManager 
Die Flink-Runtime wurde früher unter dem Namen Nephele entwickelt \cite{Warneke2009}. Sie verfügt über einen sogenannten Master-Knoten sowie mehrere Worker-Knoten. Der Master-Knoten, \textit{JobManager} genannt, hat mehrere Aufgaben. %Als Eingabe erhält er einen noch nicht optimierten Datenflussgraph. 

Er verwaltet und organisiert die Durchführung der im \textit{Jobgraph} spezifizierten Arbeitsschritte durch die Worker-Knoten, die sogenannten \textit{TaskManager}. 

Ähnlich wie andere Big-Data-Plattformen wird Flink in einer eigenen \textit{Java Virtual Machine} ausgeführt. Als Konsequenz hieraus ergibt sich die Herausforderung große Datenmengen innerhalb dieser virtuellen Maschine effizient zu speichern und zu verwalten. Flink verfügt über eine eigene Speicherverwaltung, die von Datenbank-Management-Systemen entliehene Technologien und das Map-Reduce Paradigma kombiniert. Sie besitzt eine fest definierte Anzahl, in der Regel 70\% des verfügbaren Speichers der virtuellen Maschine, an vorbelegten Speicherbereichen, die \textit{MemorySegment} genannt werden und vom \textit{MemoryManager} verwaltet werden. Jeder Speicherbereich besitzt ein reguläres, 32 Kilobyte großes Java Byte-Array, auf das mithilfe unsicherer Java-Methoden zugegriffen wird. Durch die Nutzung der unsicheren Java-Methoden werden die sonst üblichen Kontrollmechanismen umgangen, um eine effizientere Ausführung zu ermöglichen. Darüber hinaus sind die Speicherbereiche erneut nutzbar, ohne dass der der Speicher neu zugewiesen werden muss. Somit sinkt der Bedarf für eine automatische Speicherbereinigung massiv. Damit die Daten in einem \textit{MemorySegment} gespeichert werden können, müssen sie als binäre Datenrepräsentation vorliegen. Um auch nicht binäre Daten speichern zu können verwendet Flink ein eigenes Serialisierungsystem. Mit dessen Hilfe lässt sich die binäre Repräsentation der Daten bestimmen. Des weiteren ist es für die Bearbeitung der binären Daten zwingend erforderlich das gesamte Serialisierungsystem zu kennen. Nur wenn das der Fall ist, lassen sich die binären Daten robust verarbeiten. Flinks Serialisierung von Daten ist vom jeweiligen Datentyp abhängig. Allgemein wird zwischen atomaren und zusammengesetzten Datentypen unterschieden. Jeder in Flink definierte Datentyp besitzt ein entsprechendes \textit{TypeInformation}-Objekt. Dieses enthält unter anderem Informationen zur Größe und anderen Eigenschaften des Datentyps, beispielsweise ob es sich um einen atomaren Typ handelt oder einen individuellen Schlüssel, der für Vergleichsoperationen genutzt werden kann. Um allen von Flink genutzten Daten das zugehörige \textit{TypeInformation}-Objekt zuweisen zu können, benötigt Flink bereits vor der Ausführung des Programms Kenntnis der zu nutzenden Datentypen. Abhängig von der genutzten Programmiersprache müssen die Datentypen zusätzlich angegeben werden oder können von Flink automatisch ermittelt werden. Aufgrund der Kenntnis der genutzten Datentypen nutzt Flink einen zum zu serialisierenden Datentyp passenden Serialisierer, um binäre Datenrepräsentationen zu schaffen. Entsprechend werden bei der Umwandlung in die ursprünglichen Datenobjekte die zugehörigen Deserialisierer genutzt. Um zusammengesetzte Typen zu serialisieren beziehungsweise zu deserialisieren werden diese in die atomaren Typen zerlegt, aus denen sie bestehen. Auf diese atomaren Typen werden wiederum die jeweiligen regulären Serialisierer und Deserialisierer angewendet. Die intern binäre Repräsentation der Daten ermöglicht es Flink Datentransformationen schneller und effizienter umzusetzen. Als Basis für die Implementationen werden Algorithmen genutzt, die von relationalen Datenbank-Management-Systemen \cite{Codd1970} übernommen wurden. Für die Serialisierung und eventuell nötige Vergleichsoperationen zwischen Daten werden eigene Implementationen genutzt. \cite{Hueske2015}

Jede systemeigene Datentransformation in Flink besitzt einen individuell festgelegten Bedarf an Speicher. Dieser wird bei der Initialisierung des Transformationsalgorithmus in Form einer Menge an \textit{MemorySegments} durch den \textit{MemoryManager} bereitgestellt. Bei der Ausführung der Transformation wird vorrangig eben dieser Speicherbereich genutzt. Dieser wird bei Vergleichsoperationen in zwei Bereiche geteilt. Während der erste Bereich die binären Daten komplett enthält, besteht der zweite Bereich lediglich aus Zeigern, die auf die zugehörigen Daten im ersten Bereich verweisen, und aus möglicherweise vorhandenen, sortierbaren Schlüsseln besteht. Diese sind nicht zwingend Teil jedes Datentyps, da nicht alle Datentypen mit einer sortierbaren Ordnung kompatibel sind. Aufgrund dieser Struktur sind effiziente Sortier- und Austauschoperationen möglich, da nur die Zeiger, nicht jedoch die Daten bewegt werden müssen. Falls die elementweise ausgeführte Vergleichsoperation kein verwertbares Ergebnis erzielt werden die binären Daten deserialisiert und die daraus resultierenden Objekte werden miteinander verglichen. Dies gilt sowohl bei Gleichheit der sortierbaren Schlüssel zweier Elemente als auch wenn keine Schlüssel vorhanden sind. Aufgrund der vorrangig auf binär dargestellten Zeigern und Schlüsseln ausgeführten Operationen wird eine effiziente Ausführung von Datentransformationen ermöglicht. \cite{Hueske2015}

Der Kern eines Flink-Programms ist die \textit{ExecutionEnvironment}. Diese beinhaltet Informationen über die Ausführung des Programms und verwaltet den Datenzugriff auf externe Ressourcen. Mithilfe einer vom Dateityp abhängigen \textit{read}-Funktion der \textit{ExecutionEnvironment} lassen sich die zu analysierenden Daten in das Programm einbinden. Jede \textit{read}-Funktion gibt ein \textit{DataSet}-Objekt zurück, das die eingebundenen Daten enthält. Flink nutzt zur Verwaltung der Daten eine eigene Speicherverwaltung, die auf Basis der eingebundenen Daten und deren Eigenschaften optimiert wird. Dadurch wird der Speicherverbrauch verringert und der Aufwand einer Speicherbereinigung sinkt. Dies wirkt sich positiv auf die Geschwindigkeit der Plattform aus.

Aus der Anwendung der Flink-Operatoren auf einen Datensatz folgt eine Transformation des Datensatzes. Diese ist abhängig vom verwendeten Operator. Neben den mit dem Map-Reduce Modell eingeführten Operatoren \textit{map} und \textit{reduce} beinhaltet Flink noch weitere Operatoren, die den Funktionsumfang von Flink gegenüber traditionellen Map-Reduce Implementierungen erweitern. Beispiele hierfür sind \textit{filter}, \textit{join} und \textit{distinct}. Darüber hinaus existieren auch Operatoren, die sich auf gruppierte Datensätze anwenden lassen.

Apache Flink erweitert dabei das Map-Reduce Paradigma um weitere Operatoren. 

\begin{table}
	\begin{tabular}[c]{| l | p{14cm} |}
		\hline
		map & Def map%Der \textit{map}-Operator wendet eine nutzerdefinierte \textit{map}-Funktion auf jedes Element eines DataSets an. Diese muss genau ein Element zurückgeben.  
		\\
		\hline
		reduce & Def reduce \\
		\hline
		filter & Der \textit{filter}-Operator prüft jedes Element eines DataSets auf eine nutzdefinierte, binär auswertbare \textit{filter}-Bedingung. Alle Elemente, auf die diese Bedingung nicht zutrifft, werden aus dem DataSet entfernt. \\
		\hline
		join & def join \\
		\hline
		union & def union \\
		\hline
	\end{tabular}
	\caption{Wichtige Operatoren in Apache Flink}
\end{table}

%----------------------------------------------------------------------------------

\subsection{Python}
Python ist eine quelloffene und universell einsetzbare Programmiersprache, die seit 1989 existiert und fortwährend weiter entwickelt wird. Prägende Eigenschaften der Sprache sind unter anderem eine dynamische Typisierung von Variablen, eine simpel gehaltene Syntax und die Erweiterbarkeit durch Module und Bibliotheken. Es ist auch möglich Python-Code durch C- beziehungsweise C++-Bibliotheken zu erweitern \cite{Martelli2006}. Dies ermöglicht eine verkürzte Ausführungszeit eines Programms, insbesondere bei rechenintensiven Programmabschnitten. Ein Schwachpunkt von Python im Bezug auf die schnelle Verarbeitung großer Datenmengen ist die nicht auf automatisierte Parallelisierung ausgelegte Architektur. Daraus resultiert eine unzureichende Skalierbarkeit, sobald Daten, deren Größe die Arbeitsspeichergröße der ausführenden Maschine übersteigt, verarbeitet werden müssen. \textcolor{green}{(Auf weiter oben genannten Punkt der Großen Datenmengen eingehen). Bez. der Eignung zur Lösung solcher Probleme}. 

%----------------------------------------------------------------------------------

\chapter[Algorithmus zur Analyse von Pixelzeitreihen]{Beschreibung und Umsetzung des Algorithmus zur Analyse von Pixelzeitreihen}
\section[Beschreibung des Algorithmus]{Beschreibung des Algorithmus zur Analyse von Pixelzeitreihen}
Beschreibung der Vorgehensweise bei der Analyse (Zhu, SVR), Ziel der Analyse, Entwicklungsgeschichte der Analysetechnik
\section{Umsetzung des Algorithmus mit Apache Flink}
\subsection{Nutzung der Java-Programmierschnittstelle}
\subsection{Nutzung der Python-Programmierschnittstelle}
\section{Umsetzung des Algorithmus in Python}
%Einordnung als Big-Data Problem (Einordnung am GfZ bei GeoMultiSens)
Die Analyse von Satellitenbildern erfordert die Verarbeitung großer Mengen komplexer Rohdaten, \textcolor{green}{die nahezu in Echtzeit verfügbar sind}. Aufgrund dieser Charakteristika handelt es sich bei dieser Analyse um ein \textcolor{blue}{Big-Data Problem}. Denn alle vier Kriterien, die ein solches charakterisieren sind erfüllt.
\newline
Bei der Analyse von Satellitenbildern sind die Merkmale Datengröße und Datenkomplexität sowie die schnelle Verarbeitung der Daten von Bedeutung. Abhängig von der Anzahl der genutzten Bilder sind die zu verarbeitenden Datenmengen sehr groß. Ein Bild besitzt im Regelfall abhängig vom Satellitenmodell, das die Aufnahme gemacht hat, eine Größe von 750 Megabyte bis zu 1,5 Gigabyte. Um eine Entwicklung zu untersuchen werden jedoch viele dieser Bilder in die Untersuchung mit einbezogen, so dass die zu verarbeitende Datenmenge kontinuierlich ansteigt. Dieser kontinuierliche Anstieg entsteht dadurch, dass aktuell mehrere Satelliten mit der Fernerkundung der Erde fortfahren und so in kurzen Intervallen neue Bilder zur Verfügung stehen, die im Rahmen der Analyse verwendet werden sollen. \textcolor{green}{Quelle}.
%Big Data + Eigenschaften,



%Datenstrom mit Flink Operatoren
\begin{tikzpicture}[>=latex']
        \tikzset{block/.style= {draw, rectangle, align=center,minimum width=2cm,minimum height=1cm},
        rblock/.style={draw, shape=rectangle,rounded corners=1.5em,align=center,minimum width=2cm,minimum height=1cm},
        input/.style={ % requires library shapes.geometric
        draw,
        trapezium,
        trapezium left angle=60,
        trapezium right angle=120,
        minimum width=1cm,
        align=center,
        minimum height=1cm
    },
        }
        \node [rblock]  (start) {Input data};
        %Filter the data
        \node [block, below =1cm of start, label={[name=l] Filter the data}, draw] (filterDataInner) {filter('valid')};
        \node [fit=(filterDataInner) (l), draw] (filterDataOuter) {};
        %Group the data
        \node [block, below =1cm of filterDataOuter, label={[name=l] Group the filtered data}, draw] (groupDataInner) {groupBy('geographic position(maybe as a tile)')};
        \node [fit=(groupDataInner) (l), draw] (groupDataOuter) {};
        %Sort every grouped dataSet
        \node [block, below =1cm of groupDataOuter, label={[name=l] Sort every grouped dataSet}, draw] (sortDataInner) {sortPartition('time')};
        \node [fit=(sortDataInner) (l), draw] (sortDataOuter) {};
        %Approx the missing data
        \node [block, below = 1cm of sortDataOuter, label={[name=l] Approx. the missing data}, draw] (approximateDataInner) {groupReduce()};
        \node [fit=(approximateDataInner) (l), draw] (approximateDataOuter) {};
        %Predict the future data
        \node [block, below = 1cm of approximateDataOuter, label={[name=l] Predict the future data}, draw] (predictDataInner) {groupReduce()};
        \node [fit=(predictDataInner) (l), draw] (predictDataOuter) {};
        %Sink the data
        \node [rblock, below = 1cm of predictDataOuter, draw] (end) {Sink};

        \node [coordinate, below right =1cm and 1cm of start] (right) {};  %% Coordinate on right and middle
        \node [coordinate,above left =1cm and 1cm of start] (left) {};

%% paths
        \path[draw,->] (start) edge (filterDataOuter)
                    (filterDataOuter) edge (groupDataOuter)
                    (groupDataOuter) edge (sortDataOuter)
                    (sortDataOuter) edge (approximateDataOuter)
                    (approximateDataOuter) edge (predictDataOuter)
                    (predictDataOuter) edge (end)
                    ;
\end{tikzpicture}

%Minimal diagram example
%\begin{tikzpicture}[nodes=draw]
%    \node [label=label1,draw] (node1) {Node1};
%\end{tikzpicture}

%----------------------------------------------------------------------------------

\chapter{Evaluierung}
%Korrektheit, Performant, welche Impl. anderen überlegen [bei welchen Gesichtspunkten]
\section{Versuchsbeschreibung}
Beschreibung + Begründung für meine Versuchsbedingungen
\section{Auswertung}
Beschreibung und Bewertung der Ergebnisse meiner Untersuchungen

\chapter{Fazit}
Fazit und Ausblick
z.b. Vergleich mit anderen Untersuchungen

